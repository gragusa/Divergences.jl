% Options for packages loaded elsewhere
\PassOptionsToPackage{unicode}{hyperref}
\PassOptionsToPackage{hyphens}{url}
\PassOptionsToPackage{dvipsnames,svgnames,x11names}{xcolor}
%
\documentclass[
  letterpaper,
  DIV=11,
  numbers=noendperiod]{scrartcl}

\usepackage{amsmath,amssymb}
\usepackage{iftex}
\ifPDFTeX
  \usepackage[T1]{fontenc}
  \usepackage[utf8]{inputenc}
  \usepackage{textcomp} % provide euro and other symbols
\else % if luatex or xetex
  \usepackage{unicode-math}
  \defaultfontfeatures{Scale=MatchLowercase}
  \defaultfontfeatures[\rmfamily]{Ligatures=TeX,Scale=1}
\fi
\usepackage{lmodern}
\ifPDFTeX\else  
    % xetex/luatex font selection
\fi
% Use upquote if available, for straight quotes in verbatim environments
\IfFileExists{upquote.sty}{\usepackage{upquote}}{}
\IfFileExists{microtype.sty}{% use microtype if available
  \usepackage[]{microtype}
  \UseMicrotypeSet[protrusion]{basicmath} % disable protrusion for tt fonts
}{}
\makeatletter
\@ifundefined{KOMAClassName}{% if non-KOMA class
  \IfFileExists{parskip.sty}{%
    \usepackage{parskip}
  }{% else
    \setlength{\parindent}{0pt}
    \setlength{\parskip}{6pt plus 2pt minus 1pt}}
}{% if KOMA class
  \KOMAoptions{parskip=half}}
\makeatother
\usepackage{xcolor}
\setlength{\emergencystretch}{3em} % prevent overfull lines
\setcounter{secnumdepth}{-\maxdimen} % remove section numbering
% Make \paragraph and \subparagraph free-standing
\makeatletter
\ifx\paragraph\undefined\else
  \let\oldparagraph\paragraph
  \renewcommand{\paragraph}{
    \@ifstar
      \xxxParagraphStar
      \xxxParagraphNoStar
  }
  \newcommand{\xxxParagraphStar}[1]{\oldparagraph*{#1}\mbox{}}
  \newcommand{\xxxParagraphNoStar}[1]{\oldparagraph{#1}\mbox{}}
\fi
\ifx\subparagraph\undefined\else
  \let\oldsubparagraph\subparagraph
  \renewcommand{\subparagraph}{
    \@ifstar
      \xxxSubParagraphStar
      \xxxSubParagraphNoStar
  }
  \newcommand{\xxxSubParagraphStar}[1]{\oldsubparagraph*{#1}\mbox{}}
  \newcommand{\xxxSubParagraphNoStar}[1]{\oldsubparagraph{#1}\mbox{}}
\fi
\makeatother

\usepackage{color}
\usepackage{fancyvrb}
\newcommand{\VerbBar}{|}
\newcommand{\VERB}{\Verb[commandchars=\\\{\}]}
\DefineVerbatimEnvironment{Highlighting}{Verbatim}{commandchars=\\\{\}}
% Add ',fontsize=\small' for more characters per line
\usepackage{framed}
\definecolor{shadecolor}{RGB}{241,243,245}
\newenvironment{Shaded}{\begin{snugshade}}{\end{snugshade}}
\newcommand{\AlertTok}[1]{\textcolor[rgb]{0.68,0.00,0.00}{#1}}
\newcommand{\AnnotationTok}[1]{\textcolor[rgb]{0.37,0.37,0.37}{#1}}
\newcommand{\AttributeTok}[1]{\textcolor[rgb]{0.40,0.45,0.13}{#1}}
\newcommand{\BaseNTok}[1]{\textcolor[rgb]{0.68,0.00,0.00}{#1}}
\newcommand{\BuiltInTok}[1]{\textcolor[rgb]{0.00,0.23,0.31}{#1}}
\newcommand{\CharTok}[1]{\textcolor[rgb]{0.13,0.47,0.30}{#1}}
\newcommand{\CommentTok}[1]{\textcolor[rgb]{0.37,0.37,0.37}{#1}}
\newcommand{\CommentVarTok}[1]{\textcolor[rgb]{0.37,0.37,0.37}{\textit{#1}}}
\newcommand{\ConstantTok}[1]{\textcolor[rgb]{0.56,0.35,0.01}{#1}}
\newcommand{\ControlFlowTok}[1]{\textcolor[rgb]{0.00,0.23,0.31}{\textbf{#1}}}
\newcommand{\DataTypeTok}[1]{\textcolor[rgb]{0.68,0.00,0.00}{#1}}
\newcommand{\DecValTok}[1]{\textcolor[rgb]{0.68,0.00,0.00}{#1}}
\newcommand{\DocumentationTok}[1]{\textcolor[rgb]{0.37,0.37,0.37}{\textit{#1}}}
\newcommand{\ErrorTok}[1]{\textcolor[rgb]{0.68,0.00,0.00}{#1}}
\newcommand{\ExtensionTok}[1]{\textcolor[rgb]{0.00,0.23,0.31}{#1}}
\newcommand{\FloatTok}[1]{\textcolor[rgb]{0.68,0.00,0.00}{#1}}
\newcommand{\FunctionTok}[1]{\textcolor[rgb]{0.28,0.35,0.67}{#1}}
\newcommand{\ImportTok}[1]{\textcolor[rgb]{0.00,0.46,0.62}{#1}}
\newcommand{\InformationTok}[1]{\textcolor[rgb]{0.37,0.37,0.37}{#1}}
\newcommand{\KeywordTok}[1]{\textcolor[rgb]{0.00,0.23,0.31}{\textbf{#1}}}
\newcommand{\NormalTok}[1]{\textcolor[rgb]{0.00,0.23,0.31}{#1}}
\newcommand{\OperatorTok}[1]{\textcolor[rgb]{0.37,0.37,0.37}{#1}}
\newcommand{\OtherTok}[1]{\textcolor[rgb]{0.00,0.23,0.31}{#1}}
\newcommand{\PreprocessorTok}[1]{\textcolor[rgb]{0.68,0.00,0.00}{#1}}
\newcommand{\RegionMarkerTok}[1]{\textcolor[rgb]{0.00,0.23,0.31}{#1}}
\newcommand{\SpecialCharTok}[1]{\textcolor[rgb]{0.37,0.37,0.37}{#1}}
\newcommand{\SpecialStringTok}[1]{\textcolor[rgb]{0.13,0.47,0.30}{#1}}
\newcommand{\StringTok}[1]{\textcolor[rgb]{0.13,0.47,0.30}{#1}}
\newcommand{\VariableTok}[1]{\textcolor[rgb]{0.07,0.07,0.07}{#1}}
\newcommand{\VerbatimStringTok}[1]{\textcolor[rgb]{0.13,0.47,0.30}{#1}}
\newcommand{\WarningTok}[1]{\textcolor[rgb]{0.37,0.37,0.37}{\textit{#1}}}

\providecommand{\tightlist}{%
  \setlength{\itemsep}{0pt}\setlength{\parskip}{0pt}}\usepackage{longtable,booktabs,array}
\usepackage{calc} % for calculating minipage widths
% Correct order of tables after \paragraph or \subparagraph
\usepackage{etoolbox}
\makeatletter
\patchcmd\longtable{\par}{\if@noskipsec\mbox{}\fi\par}{}{}
\makeatother
% Allow footnotes in longtable head/foot
\IfFileExists{footnotehyper.sty}{\usepackage{footnotehyper}}{\usepackage{footnote}}
\makesavenoteenv{longtable}
\usepackage{graphicx}
\makeatletter
\newsavebox\pandoc@box
\newcommand*\pandocbounded[1]{% scales image to fit in text height/width
  \sbox\pandoc@box{#1}%
  \Gscale@div\@tempa{\textheight}{\dimexpr\ht\pandoc@box+\dp\pandoc@box\relax}%
  \Gscale@div\@tempb{\linewidth}{\wd\pandoc@box}%
  \ifdim\@tempb\p@<\@tempa\p@\let\@tempa\@tempb\fi% select the smaller of both
  \ifdim\@tempa\p@<\p@\scalebox{\@tempa}{\usebox\pandoc@box}%
  \else\usebox{\pandoc@box}%
  \fi%
}
% Set default figure placement to htbp
\def\fps@figure{htbp}
\makeatother

\KOMAoption{captions}{tableheading}
\makeatletter
\@ifpackageloaded{caption}{}{\usepackage{caption}}
\AtBeginDocument{%
\ifdefined\contentsname
  \renewcommand*\contentsname{Table of contents}
\else
  \newcommand\contentsname{Table of contents}
\fi
\ifdefined\listfigurename
  \renewcommand*\listfigurename{List of Figures}
\else
  \newcommand\listfigurename{List of Figures}
\fi
\ifdefined\listtablename
  \renewcommand*\listtablename{List of Tables}
\else
  \newcommand\listtablename{List of Tables}
\fi
\ifdefined\figurename
  \renewcommand*\figurename{Figure}
\else
  \newcommand\figurename{Figure}
\fi
\ifdefined\tablename
  \renewcommand*\tablename{Table}
\else
  \newcommand\tablename{Table}
\fi
}
\@ifpackageloaded{float}{}{\usepackage{float}}
\floatstyle{ruled}
\@ifundefined{c@chapter}{\newfloat{codelisting}{h}{lop}}{\newfloat{codelisting}{h}{lop}[chapter]}
\floatname{codelisting}{Listing}
\newcommand*\listoflistings{\listof{codelisting}{List of Listings}}
\makeatother
\makeatletter
\makeatother
\makeatletter
\@ifpackageloaded{caption}{}{\usepackage{caption}}
\@ifpackageloaded{subcaption}{}{\usepackage{subcaption}}
\makeatother

\usepackage{bookmark}

\IfFileExists{xurl.sty}{\usepackage{xurl}}{} % add URL line breaks if available
\urlstyle{same} % disable monospaced font for URLs
\hypersetup{
  pdftitle={Divergences},
  colorlinks=true,
  linkcolor={blue},
  filecolor={Maroon},
  citecolor={Blue},
  urlcolor={Blue},
  pdfcreator={LaTeX via pandoc}}


\title{Divergences}
\author{}
\date{}

\begin{document}
\maketitle


\section{\texorpdfstring{\texttt{Divergences.el}}{Divergences.el}}\label{divergences.el}

\texttt{Divergences} is a Julia package that makes it easy to evaluate
the value of divergences and their derivatives.

\subsection{Definition}\label{definition}

A divergence between \(a\in \mathbb{R}^n\) and \(b\in\mathbb{R}^n\) is
defined as \[
D(a,b) = \sum_{i=1}^n \gamma(a_i/b_i) b_i,
\] where \(\gamma:(a_{\gamma},+\infty)\to\mathbb{R}_{+}\),
\(a_{\gamma}\in\mathbb{R}\) is strictly convex and twice continuously
differentiable on the interior of its domain. The divergence function
\(\gamma\) is normalized as to satisfy \(\gamma(1) = 0\),
\(\gamma'(1)=0\), and \(\gamma''(1)=1\).

The gradient and the hessian of the divergence with respect to \(a\) are
given by \[
\nabla_{a}D(a,b)\equiv\left.\frac{\partial D(u,v)}{\partial u}\right|_{u=a,v=b}=\begin{pmatrix}\gamma'(a_{1}/b_{1})\\
\gamma'(a_{2}/b_{2})\\
\vdots\\
\gamma'(a_{n}/b_{n})
\end{pmatrix}\] and \[
\nabla_{a}^{2}D(a,b)\equiv\left.\frac{\partial^{2}D(u,v)}{\partial u\partial u}\right|_{u=a,v=b}=\begin{pmatrix}\frac{\gamma''(a_{1}/b_{1})}{b_{1}} & 0 & \cdots & 0\\
0 & \frac{\gamma''(a_{2}/b_{2})}{b_{2}} & 0 & \vdots\\
\vdots & 0 & \ddots & 0\\
0 & \cdots & 0 & \frac{\gamma''(a_{n}/b_{n})}{b_{n}}
\end{pmatrix}
\] respectively. Given the normalization \(\gamma'(1)=0\), and
\(\gamma''(1)=1\), we have that \[
\nabla_{a}D(a,a) = 0, \quad \nabla^2_{a}D(a,a) = 1.
\]

The divergences implemented in the packges are given in the table below
together with their first and second order derivatives.

\begin{longtable}[]{@{}
  >{\raggedright\arraybackslash}p{(\linewidth - 8\tabcolsep) * \real{0.1687}}
  >{\raggedright\arraybackslash}p{(\linewidth - 8\tabcolsep) * \real{0.3735}}
  >{\raggedright\arraybackslash}p{(\linewidth - 8\tabcolsep) * \real{0.1145}}
  >{\raggedright\arraybackslash}p{(\linewidth - 8\tabcolsep) * \real{0.2289}}
  >{\raggedright\arraybackslash}p{(\linewidth - 8\tabcolsep) * \real{0.1145}}@{}}
\toprule\noalign{}
\begin{minipage}[b]{\linewidth}\raggedright
\textbf{Divergence}
\end{minipage} & \begin{minipage}[b]{\linewidth}\raggedright
\textbf{\(\gamma(u)\)}
\end{minipage} & \begin{minipage}[b]{\linewidth}\raggedright
Domain
\end{minipage} & \begin{minipage}[b]{\linewidth}\raggedright
\textbf{\(\nabla_\gamma(u)\)}
\end{minipage} & \begin{minipage}[b]{\linewidth}\raggedright
\textbf{\(H_\gamma(u)\)}
\end{minipage} \\
\midrule\noalign{}
\endhead
\bottomrule\noalign{}
\endlastfoot
\textbf{Kullback-Leibler} & \(u \log(u) - u + 1\) & \((0,+\infty)\) &
\(\log(u)\) & \(1/u\) \\
\textbf{Reverse Kullback-Leibler} & \(\log(u) + u - 1\) &
\((0,+\infty)\) & \(-\frac{1}{u} + 1\) & \(\frac{1}{u^2}\) \\
\textbf{Hellinger} & \(2u + (2 - 4\sqrt{u})\) & \((0,+\infty)\) &
\(2\left(1 - \frac{1}{\sqrt{u}}\right)\) & \(\frac{1}{u^{3/2}}\) \\
\textbf{Chi-Squared} & \(\frac{1}{2}(u - 1)^2\) & \((-\infty,+\infty)\)
& \(u - 1\) & \(1\) \\
\textbf{Cressie-Read} &
\(\frac{u^{1+\alpha} + \alpha - u(1+\alpha)}{\alpha(1+\alpha)}\) &
\((0,+\infty)\) & \(\frac{u^\alpha - 1}{\alpha}\) & \(u^{\alpha-1}\) \\
\end{longtable}

The convex conjugate conjugate of \(\gamma\) is defined as \[
\gamma^*(u) = \sup_{u\in\mathbb{R}} \left\{u\upsilon - \gamma(u)\right\}.
\] For continuously twice differentiable function, the convex conjugate
is \[
\gamma^*(z) = (\gamma')^{-1}(z) \cdot z - \gamma\left((\gamma')^{-1}(z)\right).
\] where \((\gamma')^{-1}(z) := {u: \gamma'(x) = z}\). The domain of
\(\gamma^*\) is \((-\infty, d)\), where \[
d = \lim_{u\to +\infty} \gamma(u)/u.
\]

\subsection{Divergences}\label{divergences}

The Cressie Read is a family of divergences. Members of this family are
indexed a parameter \(\alpha\). This family contains the chi-squared
divergence (\(\alpha = 1\)), the Kullback Leibler divergence
(\(a \to 0\)), the reverse Kullback Leibler divergence (\(a \to -1\)),
and the Hellinger distance (\(a = -1/2\)).

Since if \(\alpha<0\), \(\gamma\) in the Cressie Read family is not
convex on \((-\infty 0)\) and thus we set \(\gamma(u)=+\infty\).

\subsection{Modified divergences}\label{modified-divergences}

\begin{longtable}[]{@{}
  >{\raggedright\arraybackslash}p{(\linewidth - 6\tabcolsep) * \real{0.1657}}
  >{\raggedright\arraybackslash}p{(\linewidth - 6\tabcolsep) * \real{0.2571}}
  >{\raggedright\arraybackslash}p{(\linewidth - 6\tabcolsep) * \real{0.2571}}
  >{\raggedright\arraybackslash}p{(\linewidth - 6\tabcolsep) * \real{0.3200}}@{}}
\toprule\noalign{}
\begin{minipage}[b]{\linewidth}\raggedright
\textbf{Divergence}
\end{minipage} & \begin{minipage}[b]{\linewidth}\raggedright
\textbf{\(\gamma^*(\theta, b)\)}
\end{minipage} & \begin{minipage}[b]{\linewidth}\raggedright
\textbf{\(\lim_{u \to \infty} \frac{\gamma(u)}{u}\)}
\end{minipage} & \begin{minipage}[b]{\linewidth}\raggedright
\textbf{\(\lim_{u \to \infty} \frac{u \gamma'(u)}{\gamma(u)}\)}
\end{minipage} \\
\midrule\noalign{}
\endhead
\bottomrule\noalign{}
\endlastfoot
\textbf{Kullback-Leibler} & \(b (e^\theta - 1)\) & \(\log b - 1\) &
\(1\) \\
\textbf{Reverse Kullback-Leibler} & \(b \log(1 - \theta) + b\),
\(\theta < 1\) & \(1\) & \(1\) \\
\textbf{Hellinger} & \(b (1 - 2\sqrt{1 - \theta})\), \(\theta \leq 1\) &
\(2\) & \(0\) \\
\textbf{Chi-Squared} & \(b\left(\theta + \frac{\theta^2}{2}\right)\) &
\(\infty\) & \(2\) \\
\textbf{Cressie-Read} & Depends on \(\alpha\) & Depends on \(\alpha\) &
Depends on \(\alpha\) \\
\textbf{Modified Divergence} & Derived from
\(\gamma_0, \gamma_1, \gamma_2\) & Depends on parameters & Depends on
parameters \\
\textbf{Fully Modified Divergence} & Depends on
\(\gamma_U, \gamma_L, \rho, \phi\) & Depends on \(\rho, \phi\) & Depends
on \(\rho, \phi\) \\
\end{longtable}

For many of the divergences defined above the effective domain of their
conjugate, \(\gamma^*\), does not span \(\mathbb{R}\) since
\(\gamma(u)/u \to l < +\infty\) as \(u \to +\infty\).

For some \(\vartheta>0\), let \(u_{\vartheta}\equiv 1+\vartheta\). The
modified divergence \(\gamma_{\vartheta}\) is defined as \[
\gamma_{\vartheta}(u) = \begin{cases}
  \gamma(u_{\vartheta}) + \gamma'(u_{\vartheta})(u-u_{\vartheta}) + \frac{1}{2}\gamma''(u_{\vartheta})(u-u_{\vartheta})^2, & u\geqslant u_{\vartheta}\\
  \newline\gamma(u), & u\in (0,u_{\vartheta})\\
  \newline \lim_{u\to 0^{+}} \gamma(u), & u=0 \\
  \newline+\infty, &  u<0
\end{cases}.
\]

It is immediate to verify that this divergence still satisfies all the
requirements and normalization of \(\gamma\). Furthermore, it holds that
\[
  \lim_{u\to\infty}\frac{\gamma_{\vartheta}(u)}{u} = +\infty,
    \qquad \text{and}\qquad
  \lim_{u\to\infty}\frac{u\gamma'_{\vartheta}(u)}{\gamma_{\vartheta}(u)} = 2.
\]

The first limit implies that the image of \(\gamma'_{\vartheta}\) is the
real line and thus
\(\overline{\mathrm{dom}\,\gamma^*_{\vartheta}}=(-\infty,+\infty)\). The
expression for the conjugate is obtained by applying the
Legendre-Fenchel transform to obtain \[
\gamma_{\vartheta}^*(u) =
\begin{cases}
  a_{\vartheta}\upsilon^2 + b_{\vartheta}\upsilon + c_{\vartheta}, & \upsilon>\gamma'(u_{\vartheta}),\\
  \newline \gamma^*(\upsilon), & u\leqslant \gamma'(u_{\vartheta})
\end{cases},
\]

where \(a_{\vartheta} = 1/(2\gamma''(u_{\vartheta}))\),
\(b_{\vartheta}=u_{\vartheta} - 2a_{\vartheta}\gamma'(u_{\vartheta})\),
and \(c_{\vartheta}=-\gamma(u_{\vartheta}) +
a_{\vartheta}\gamma'(u_{\vartheta}) - u_{\vartheta}^2/a_{\vartheta}\).
The conjugate \(\gamma_{\vartheta}^*(u)\) will have a closed form
expression when so does the original divergence function.

\subsection{Fully modified
divergences}\label{fully-modified-divergences}

For some \(\vartheta>0\) and \(0 < \varphi < 1-a_{\gamma}\), let
\(u_{\vartheta}\equiv
1+\vartheta\) and \(u_{\varphi} = a_{\gamma} + \varphi\). The
\textbf{fully} modified divergence \(\gamma_{\varphi, \vartheta}\) is
defined as \[
\gamma_{\vartheta}(u) = \begin{cases}
  \gamma(u_{\vartheta}) + \gamma'(u_{\vartheta})(u-u_{\vartheta}) + \frac{1}{2}\gamma''(u_{\vartheta})(u-u_{\vartheta})^2, & u\geqslant u_{\vartheta}\\
  \newline\gamma(u), & u\in (u_{\varphi},u_{\vartheta})\\
  \newline    \gamma(u_{\varphi}) + \gamma'(u_{\varphi})(u-u_{\varphi}) + \frac{1}{2}\gamma''(u_{\varphi})(u-u_{\varphi})^2, & u\leqslant u_{\varphi}\\
\end{cases}.
\] It is immediate to verify that this divergence still satisfies all
the requirements and normalization of \(\gamma\), while being defined on
all \(\mathbb{R}\).

\subsection{Example of divergences}\label{example-of-divergences}

The following divergence types are defined by \texttt{Divergences}.

\paragraph{Kullback-Leibler
divergence}\label{kullback-leibler-divergence}

\[
D^{KL}(a,b) = \sum_{i=1}^n \gamma^{KL}(a_i/b_i) b_i
\]

\[
\gamma^{KL}(u) = u\log(u) - u + 1
\]

The gradient and the hessian are given by

\[
\nabla_{a}^{2}D^{KL}(a,b) = \left(\log(a_1/b_1),\ldots,\log(a_n,b_n)
\right), \quad \nabla_{a}^{2}D^{KL}(a,b) = \mathrm{diag}(1/a_1, \ldots, 1/a_n)
\]

\paragraph{Reverse Kullback-Leibler
divergence}\label{reverse-kullback-leibler-divergence}

\[
D^{rKL}(a,b) = \sum_{i=1}^n \gamma^{rKL}(a_i/b_i) b_i
\]

\[
\gamma^{rKL}(u) = -\log(u) + u - 1
\]

The gradient and the hessian are given by

\[
\nabla_{a}^{2}D^{rKL}(a,b) = \left(1-b_1/a_1,\ldots, 1 - b_n/a_n
\right), \quad \nabla_{a}^{2}D^{rKL}(a,b) = \mathrm{diag}(b_1/a^2_1, \ldots, b_n/a^2_n)
\]

For reverse Kullback Leibler divergence, \(\gamma(u)=-\log(u)+u-1\), we
have that \(\gamma(u)/u \to 0\) as \(u\to\infty\). The modified reverse
Kullback Leibler divergence is given by \[
    \gamma_{\vartheta}(u) =
    \begin{cases}
      -\log(u_{\vartheta}) + (1-\frac{1}{u_{\vartheta}})u+ \frac{1}{2u_{\vartheta}^2}(u-u_{\vartheta})^2, &  u>u_{\vartheta}\\
      \newline -\log(u) + u - 1, &0 < u\leqslant u_{\vartheta}\\
      \newline +\infty, & u\leqslant0.
    \end{cases}.
\]

The conjugate of \(\gamma_{\theta}\) is given by \[
    \gamma_{\vartheta}(u) =
    \begin{cases}
      a_{\vartheta}\upsilon^2 + b_{\vartheta}\upsilon + c_{\vartheta}, & \upsilon > 1-\frac{1}{u_{\vartheta}} \\
    \newline -\log(1- \upsilon), & \upsilon \leqslant 1-\frac{1}{u_{\vartheta}},
    \end{cases}
\] where \(a_{\vartheta}=u^2_{\vartheta}/2\),
\(b_{\vartheta}=u_{\vartheta}(2-u_{\vartheta})\), and
\(c_{\vartheta}=\log(u_{\vartheta})-u_{\vartheta}-1+u_{\vartheta}(u_{\vartheta}-1)/2\).

\paragraph{Chi-squared divergence}\label{chi-squared-divergence}

\[
D^{\chi}(a,b) = \sum_{i=1}^n \gamma^{\chi}(a_i/b_i) b_i
\]

\[
\gamma^{\chi}(u) = u^2/2 - u + 0.5
\]

The gradient and the hessian are given by

\[
\nabla_{a}^{2}D^{\chi}(a,b) = \left((a_1 - b_1)/b_1^2, \ldots, (a_n - b_n)/b_n^2
\right), \quad \nabla_{a}^{2}D^{\chi}(a,b) =
\mathrm{diag}\left(\frac{1}{b_1^2},\ldots, \frac{1}{b_n^2}\right)
\]

\subsection{Cressie-Read divergences}\label{cressie-read-divergences}

The type \texttt{CressieRead} is a family of divergences. Members of
this family are indexed by a function \(\gamma\) indexed by parameter
\(\alpha\):

\[
\gamma_{\alpha}^{CR}(a,b)=\frac{\left(\frac{a}{b}\right)^{1+\alpha}-1}{\alpha(\alpha+1)}-\frac{\left(\frac{a}{b}\right)-1}{\alpha}.
\]

The gradient and the hessian are given by

\[
\nabla_{a}^{2}D^{CR}_{\alpha}(a,b) = \left(
\frac{\left(\frac{a_1}{b_1}\right)^{\alpha }-1}{\alpha  b_1}, \ldots,\frac{\left(\frac{a_n}{b_n}\right)^{\alpha }-1}{\alpha  b_n}
\right), \quad 
\nabla_{a}^{2}D^{CR}_{\alpha}(a,b) = \mathrm{diag}\left(\frac{\left(\frac{a_1}{b_1}\right)^{\alpha }}{a_1 b_1},\ldots,
\frac{\left(\frac{a_n}{b_n}\right)^{\alpha }}{a_n b_n}
\right)
\]

The Cressie-Read family contains the chi-squared divergence (\(\alpha =
1\)), the Kullback Leibler divergence (\(a \to 0\)), the reverse
Kullback Leibler divergence (\(a \to -1\)), and the Hellinger distance
(\(a = -1/2\)).

For instance, for the Cressie Read family of divergences defined below,
\[
\lim_{u\to +\infty}\gamma^{CR}_{\alpha}(u)/u = -1/\alpha
\] for all \(\alpha\leqslant 0\). Also, for all \(\alpha\leqslant 0\),
the divergence is not convex on \((-\infty, 0)\) and thus a fully
modified version can be considered.

\subsection{\texorpdfstring{Using \texttt{Divergences}
package}{Using Divergences package}}\label{using-divergences-package}

\begin{Shaded}
\begin{Highlighting}[]
\ImportTok{using} \BuiltInTok{Divergences}
\end{Highlighting}
\end{Shaded}

Suppose \(a = [0.2, 0.4, 0.4]\) and \(b = [0.1, 0.3, 0.6]\).

\begin{Shaded}
\begin{Highlighting}[]
\NormalTok{a }\OperatorTok{=}\NormalTok{ [}\FloatTok{0.2}\NormalTok{, }\FloatTok{0.4}\NormalTok{, }\FloatTok{0.4}\NormalTok{]}
\NormalTok{b }\OperatorTok{=}\NormalTok{ [}\FloatTok{0.1}\NormalTok{, }\FloatTok{0.3}\NormalTok{, }\FloatTok{0.6}\NormalTok{]}
\end{Highlighting}
\end{Shaded}

\texttt{\{julia;\ display=true\}\ evaluate(KullbackLeibler(),\ a,\ b)\ gradient(KullbackLeibler(),\ a,\ b)\ hessian(KullbackLeibler(),\ a,\ b)}

\texttt{\{julia;\ display=true\}\ evaluate(ReverseKullbackLeibler(),\ a,\ b)\ gradient(ReverseKullbackLeibler(),\ a,\ b)\ hessian(ReverseKullbackLeibler(),\ a,\ b)}

Yes, the convex conjugate (\gamma\^{}*(\theta)) of a
twice-differentiable function (\gamma(u)) can often be expressed in
terms of the derivatives of (\gamma(u)). Here's how:

\subsubsection{Formula for the Convex
Conjugate}\label{formula-for-the-convex-conjugate}

The convex conjugate is defined as: {[} \gamma\^{}*(\theta) =
\sup\_\{u\} \left[ u \theta - \gamma(u) \right]. {]}

For a smooth, twice-differentiable, strictly convex function
(\gamma(u)), the maximizer (u\^{}\emph{) satisfies the first-order
condition: {[} \frac{\partial}{\partial u}
\left[ u \theta - \gamma(u) \right] = 0, {]} which gives: {[} \theta =
\gamma'(u\^{}}). {]}

The corresponding convex conjugate can then be written as: {[}
\gamma\^{}\emph{(\theta) = u\^{}} \theta - \gamma(u\^{}\emph{), {]}
where (u\^{}}) is the unique solution of (\gamma'(u\^{}*) = \theta).

\subsubsection{\texorpdfstring{Expressing (\gamma\^{}*(\theta)) in Terms
of (\gamma`) and
(\gamma'\,'):}{Expressing (\^{}*()) in Terms of (`) and ('\,'):}}\label{expressing-in-terms-of-and}

\begin{enumerate}
\def\labelenumi{\arabic{enumi}.}
\item
  **Solve for (u\^{}*):** From the equation (\gamma`(u\^{}\emph{) =
  \theta), (u\^{}}) is implicitly defined as: {[} u\^{}* =
  (\gamma')\^{}\{-1\}(\theta), {]} assuming (\gamma'(u)) is invertible.
\item
  **Substitute (u\^{}*) into the conjugate:** Using the definition of
  the convex conjugate: {[} \gamma\^{}\emph{(\theta) = u\^{}} \theta -
  \gamma(u\^{}*). {]}
\item
  \textbf{Final Formula:} Substitute (u\^{}* =
  (\gamma')\^{}\{-1\}(\theta)): {[} \gamma\^{}*(\theta) =
  (\gamma`)\^{}\{-1\}(\theta) \cdot \theta -
  \gamma\left((\gamma')\^{}\{-1\}(\theta)\right). {]}
\end{enumerate}

\subsubsection{Derivative of the Convex
Conjugate:}\label{derivative-of-the-convex-conjugate}

The first derivative of the convex conjugate (\gamma\^{}\emph{(\theta))
can be found as: {[} \frac{d}{d\theta} \gamma\^{}}(\theta) =
(\gamma')\^{}\{-1\}(\theta). {]}

The second derivative can be derived using the inverse function theorem:
{[} \frac{d^2}{d\theta^2} \gamma\^{}*(\theta) =
\frac{1}{\gamma''((\gamma')^{-1}(\theta))}. {]}

\subsubsection{\texorpdfstring{Application to the Fully Modified
(\gamma\_\vartheta):}{Application to the Fully Modified (\_):}}\label{application-to-the-fully-modified-_}

For the piecewise function (\gamma\emph{\vartheta(u)): 1. In each
segment (e.g., for (u \geq u}\vartheta) or (u \leq u\_\varphi)), apply
the standard convex conjugate formula for the quadratic approximation.
2. For the middle segment (u \in (u\_\varphi, u\_\vartheta)), the
conjugate coincides with the conjugate of the base function (\gamma(u))
in that range.

The piecewise nature adds complexity, but the general approach remains
consistent.




\end{document}
